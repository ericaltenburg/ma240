%=======================02-713 LaTeX template, following the 15-210 template==================
%
% You don't need to use LaTeX or this template, but you must turn your homework in as
% a typeset PDF somehow.
%
% How to use:
%    1. Update your information in section "A" below
%    2. Write your answers in section "B" below. Precede answers for all 
%       parts of a question with the command "\question{n}{desc}" where n is
%       the question number and "desc" is a short, one-line description of 
%       the problem. There is no need to restate the problem.
%    3. If a question has multiple parts, precede the answer to part x with the
%       command "\part{x}".
%    4. If a problem asks you to design an algorithm, use the commands
%       \algorithm, \correctness, \runtime to precede your discussion of the 
%       description of the algorithm, its correctness, and its running time, respectively.
%    5. You can include graphics by using the command \includegraphics{FILENAME}
%
\documentclass[11pt]{article}
\usepackage{amsmath,amssymb,amsthm}
\usepackage{graphicx}
\usepackage[margin=1in]{geometry}
\usepackage{fancyhdr}
\newtheorem{theorem}{Theorem}
\setlength{\parindent}{0pt}
\setlength{\parskip}{5pt plus 1pt}
\setlength{\headheight}{13.6pt}
\newcommand\question[2]{\vspace{.25in}\hrule\textbf{#1}: #2\vspace{.5em}\hrule\vspace{.10in}}
\renewcommand\part[1]{\vspace{.10in}(#1)\par}
\newcommand\algorithm{\vspace{.10in}\textbf{Algorithm: }}
\newcommand\correctness{\vspace{.10in}\textbf{Correctness: }}
\newcommand\runtime{\vspace{.10in}\textbf{Running time: }}
\newcommand{\R}{\mathbb{R}}
\newcommand{\N}{\mathbb{N}}
\newcommand{\Z}{\mathbb{Z}}
\newtheorem{conjecture}[theorem]{Conjecture }
\pagestyle{fancyplain}
\lhead{\textbf{\NAME}}
\chead{\textbf{{\COURSE} Homework \HWNUM}}
\rhead{\today}
\begin{document}\raggedright
%Section A==============Change the values below to match your information==================
\newcommand\NAME{Eric Altenburg}  % your name
\newcommand\COURSE{MA-240}
\newcommand\HWNUM{5}              % the homework number
%Section B==============Put your answers to the questions below here=======================

% no need to restate the problem --- the graders know which problem is which,
% but replacing "The First Problem" with a short phrase will help you remember
% which problem this is when you read over your homeworks to study.

\textbf{Pledge:} \textit{I pledge my honor that I have abided by the Stevens Honor System.} -Eric Altenburg

\question{1}{Prove that, given any two real numbers $x$ and $y$ such that $x < y$, there exists an irrational number $z$ such that $x < z < y$.}

\begin{proof}
	Let $z$ be a real number equally distant from $x$ and $y$, $z = \frac{x+y}{2}$. To show that $x<z$, substitute $z$ for $\frac{x+y}{2}$, and
	\begin{align*}
		x &< z\\
		x &< \frac{x+y}{2}\\
		2x &< x+y\\
		x &< y.
	\end{align*} 
	Given that $x<y$, this proves the inequality to be true for $x<z$. To show that $z<y$, substitute $z$ for $\frac{x+y}{2}$ again, and 
	\begin{align*}
		z &< y\\
		\frac{x+y}{2} &< y\\
		x + y &< 2y\\
		x &< y.
	\end{align*}
	Once again, given $x<y$ this proves the inequality to be true for $z<y$.
\end{proof}

\question{2}{Let $S\subset\{1,2,\ldots,1000\}$ be a set of 100 natural numbers. Prove that there exists distinct nonempty subsets $X,Y \subset S$ such that the sum of the elements of $X$ equals the sum of the elements of $Y$.}

\begin{proof}
	The possible number of subsets of 100 natural numbers are $2^{100}$, and the largest possible sum of a subset of numbers of $S$ is $901 + 902 + \ldots + 999 + 1000 = 95050$. This means there are 95050 possible sums of numbers, and due to the Pigeonhole Principle, since there are more subsets ($2^{100}$) than there are possible sums (95050), there exists at least one sum which can be made from two subsets. 

	Let $A$ and $B$ be the two subsets that form the same sum, and let $C = A \cap B$ which are all the elements that are found in both $A$ and $B$. If we remove the set of $C$ from $A$ and $B$, then $A' = A - C$ and $B' = B-C$. Since we removed the same elements from both sets, the sum of $A'$ and $B'$ will remain the same while being distinct and non-empty. 
\end{proof}

\question{3}{Make a conjecture about which numbers $n \in \N$ can be expressed as a sum of two or more consecutive natural numbers. (Note that the numbers int he sum don't have to start at 1. For example, 12 is such a number since $12 = 3 + 4 + 5$.) Then prove your conjecture.}

\begin{conjecture}
	Every number $n \in \N$ where $n \ne 2^k$ and $k \in \N$ can be expressed as the sum of two or more consecutive natural numbers.
\end{conjecture}

\begin{proof}
	(Contradiciton) Assume a number $n$ that is a power of 2 can be written as a sum of consecutive natural numbers. The amount of numbers that can be added up to make $n$ can be an odd and even amount.

	Case 1: The summation has an odd amount of consecutive numbers.\\
	A sum of consecutive numbers having an odd amount of numbers would have one exact middle number being added together (i.e. $m + (m+1) + \ldots + (m+n)$ will have an element that is equally distant from $m$ and $(m+n)$, this is known as the average of the two numbers). Then the sum can be expressed as $\text{sum} = \text{average} \cdot \text{amount of consecutive number added together}$, the latter of which is odd. This would mean the sum has an odd number as a factor, however, a power of 2 will always be even which is a contradiction.

	Case 2: The summation has an even amount of consecutive numbers.\\
	Because the sum will have an even amount of consecutive numbers, there will not be a number that is the average like with Case 1, instead the middle two numbers must be summed and then divided by 2. This means the sum can be expressed as $\text{sum} = \text{middle two numbers summed} \cdot \frac{1}{2} \cdot \text{amount of consecutive numbers}$. We know the amount of consecutive numbers divided by 2 will still be an even number, however, two consecutive numbers added together will always form an odd number. This means the sum has an odd number as a factor, and since a power of 2 will always be even, this is a contradiction.
\end{proof}
% We know the sum of numbers from 1 to $n$ is $\frac{n(n+1)}{2}$, so the summation of numbers from $k$ to $n$ is 





	
\end{document}