%=======================02-713 LaTeX template, following the 15-210 template==================
%
% You don't need to use LaTeX or this template, but you must turn your homework in as
% a typeset PDF somehow.
%
% How to use:
%    1. Update your information in section "A" below
%    2. Write your answers in section "B" below. Precede answers for all 
%       parts of a question with the command "\question{n}{desc}" where n is
%       the question number and "desc" is a short, one-line description of 
%       the problem. There is no need to restate the problem.
%    3. If a question has multiple parts, precede the answer to part x with the
%       command "\part{x}".
%    4. If a problem asks you to design an algorithm, use the commands
%       \algorithm, \correctness, \runtime to precede your discussion of the 
%       description of the algorithm, its correctness, and its running time, respectively.
%    5. You can include graphics by using the command \includegraphics{FILENAME}
%
\documentclass[11pt]{article}
\usepackage{amsmath,amssymb,amsthm}
\usepackage{graphicx}
\usepackage[margin=1in]{geometry}
\usepackage{fancyhdr}
\newtheorem{theorem}{Theorem}
\setlength{\parindent}{0pt}
\setlength{\parskip}{5pt plus 1pt}
\setlength{\headheight}{13.6pt}
\newcommand\question[2]{\vspace{.25in}\hrule\textbf{#1}: #2\vspace{.5em}\hrule\vspace{.10in}}
\renewcommand\part[1]{\vspace{.10in}(#1)\par}
\newcommand\algorithm{\vspace{.10in}\textbf{Algorithm: }}
\newcommand\correctness{\vspace{.10in}\textbf{Correctness: }}
\newcommand\runtime{\vspace{.10in}\textbf{Running time: }}
\newcommand{\R}{\mathbb{R}}
\newcommand{\N}{\mathbb{N}}
\newcommand{\Z}{\mathbb{Z}}
\newcommand{\Q}{\mathbb{Q}}
\pagestyle{fancyplain}
\lhead{\textbf{\NAME}}
\chead{\textbf{{\COURSE} Homework \HWNUM}}
\rhead{\today}
\begin{document}\raggedright
%Section A==============Change the values below to match your information==================
\newcommand\NAME{Eric Altenburg}  % your name
\newcommand\COURSE{MA-240}
\newcommand\HWNUM{1}              % the homework number
%Section B==============Put your answers to the questions below here=======================

% no need to restate the problem --- the graders know which problem is which,
% but replacing "The First Problem" with a short phrase will help you remember
% which problem this is when you read over your homeworks to study.

\textbf{Pledge:} \textit{I pledge my honor that I have abided by the Stevens Honor System.} -Eric Altenburg

\question{1}{Prove that each of the following relations is an equivalence relation. Then describe the corresponding equivalence classes, e.g. by giving a geometric description.}

\part{The relation $R$ defined on $\R^2$ by $((a,b),(c,d)) \in R$ if $|a| + |b| = |c| + |d|.$}

\part{The relation $S$ defined on the set of positive rational numbers $\Q_{>0}$ by $(a,b) \in S \text{ if } \frac{a}{b}= 2^n$ for some $n \in \Z$.}

\question{2}{Let $R$ and $S$ be equivalence relations on a set $A$. Prove or disprove the following statements.}

\part{The relation $R \cup S$ is an equivalence relation on $A$.}

\part{The relation $R \cap S$ is an equivalence relation on $A$.}

\question{3}{The set of integers \textit{modulo n}, where $n > 1$ is a natural number, is denoted $\Z/n\Z$ and is defined as the set of equivalence classes under the equivalence relation on $\Z$ of being congruent modulo $n$. Prove that is it possible to define addition and multiplication operations on $\Z/n\Z$ via the formulas $[a]+[b] = [ a + b]$ and $[a] \cdot [b] = [ a \cdot b]$, respectively.}

	
\end{document}