%=======================02-713 LaTeX template, following the 15-210 template==================
%
% You don't need to use LaTeX or this template, but you must turn your homework in as
% a typeset PDF somehow.
%
% How to use:
%    1. Update your information in section "A" below
%    2. Write your answers in section "B" below. Precede answers for all 
%       parts of a question with the command "\question{n}{desc}" where n is
%       the question number and "desc" is a short, one-line description of 
%       the problem. There is no need to restate the problem.
%    3. If a question has multiple parts, precede the answer to part x with the
%       command "\part{x}".
%    4. If a problem asks you to design an algorithm, use the commands
%       \algorithm, \correctness, \runtime to precede your discussion of the 
%       description of the algorithm, its correctness, and its running time, respectively.
%    5. You can include graphics by using the command \includegraphics{FILENAME}
%
\documentclass[11pt]{article}
\usepackage{amsmath,amssymb,amsthm}
\usepackage{graphicx}
\usepackage[margin=1in]{geometry}
\usepackage{fancyhdr}
\newtheorem{theorem}{Theorem}
\setlength{\parindent}{0pt}
\setlength{\parskip}{5pt plus 1pt}
\setlength{\headheight}{13.6pt}
\newcommand\question[2]{\vspace{.25in}\hrule\textbf{#1}: #2\vspace{.5em}\hrule\vspace{.10in}}
\renewcommand\part[1]{\vspace{.10in}\textbf{(#1)}\par}
\newcommand\algorithm{\vspace{.10in}\textbf{Algorithm: }}
\newcommand\correctness{\vspace{.10in}\textbf{Correctness: }}
\newcommand\runtime{\vspace{.10in}\textbf{Running time: }}
\newcommand{\R}{\mathbb{R}}
\newcommand{\N}{\mathbb{N}}
\newcommand{\Z}{\mathbb{Z}}
\pagestyle{fancyplain}
\lhead{\textbf{\NAME}}
\chead{\textbf{{\COURSE} Homework \HWNUM}}
\rhead{\today}
\begin{document}\raggedright
%Section A==============Change the values below to match your information==================
\newcommand\NAME{Eric Altenburg}  % your name
\newcommand\COURSE{MA-240}
\newcommand\HWNUM{2 Corrections}              % the homework number
%Section B==============Put your answers to the questions below here=======================

% no need to restate the problem --- the graders know which problem is which,
% but replacing "The First Problem" with a short phrase will help you remember
% which problem this is when you read over your homeworks to study.

\textbf{Pledge:} \textit{I pledge my honor that I have abided by the Stevens Honor System.} -Eric Altenburg

\question{1}{Write a proof for the following theorem:}

\begin{theorem}
	Let $n \in \Z$. Then $2n^2+n$ is odd if and only if $\cos(\frac{n \pi}{2})$ is even.
\end{theorem}

\begin{proof}
	Because of the nature of the \textit{if and only if} it must be proven bidirectionally. First, consider: $2n^2+n$ is odd $\Rightarrow$ $\cos(\frac{n \pi}{2})$ is even.

	Case 1: $n$ is odd.

	Then $n=2m+1$, where $m \in \Z$, and
	\begin{align*}
		2n^2+n &= 2(2m+1)^2 + (2m+1)\\
		&= 2(4m^2+4m+1) + (2m+1)\\
		&= 8m^2+8m+2+2m+1\\
		&= 8m^2+10m+2+1\\
		&= 2(4m^2+5m+1) + 1.
	\end{align*}

	This an odd number by definition.

	Case 2: $n$ is even.

	Then $n=2m$, where $m \in \Z$, and
	\begin{align*}
		2n^2+n &= 2(2m)^2 + (2m)\\
		&= 2(4m^2)+2m\\
		&= 8m^2+2m\\
		&= 2(4m^2+m).
	\end{align*}

	This is an even number by definition. Since we want $2n^2+n$ to be odd for this implication, we can see that $n$ must be odd as well. Now let $n = 2m+1$ where $m \in \Z$. Then, 

	\begin{align*}
		\cos\left(\frac{n\pi}{2}\right) &= \cos\left(\frac{(2m+1)\pi}{2}\right)\\
		&= \cos\left(\frac{2m\pi + \pi}{2}\right)\\
		&= \cos\left(m\pi + \frac{\pi}{2}\right).
	\end{align*}

	This evaluates to 0 for any $m \in \Z$.

	The other direction of the implication that needs to be proven is: $\cos(\frac{n\pi}{2})$ is even $\Rightarrow$ $2n^2+n$ is odd. Once again if $\cos(\frac{n\pi}{2})$ is even, then it must evaluate to 0 which means $n$ must be odd. And because of case 1 previously shown, it is known that $2n^2+n$ will always be odd as well.
\end{proof}

\pagebreak 
\question{2}{Write a proof for the following theorem:}

\begin{theorem}
	Let $x,y \in \Z$. If $xy$ is odd, then $x^2 + y^2$ is even.
\end{theorem}

\begin{proof}
	Because $xy$ is odd, $x$ and $y$ must be odd numbers. To show this, we can create cases where $x$ and $y$ are different combination of even and odd numbers.

	Case 1: $x, y$ are both odd.

	Then $x=2m+1$ and $y=2n+1$, where $m,n \in \Z$, and
	\begin{align*}
		xy &= (2m+1)(2n+1)\\
		&= 4mn + 2m + 2n + 1\\
		&= 2 (2mn + m + n) + 1.
	\end{align*}
	This is the definition of an odd number.

	Case 2: $x$ is odd and $y$ is even.
	
	Then $x=2m+1$ and $y=2n$, where $m,n \in \Z$, and
	\begin{align*}
		xy &= (2m+1)(2n)\\
		&= 2(2mn+n).
	\end{align*}
	This is the definition of an even number.
	
	Case 3: $x$ is even and $y$ is odd.
	
	The proof is identical to the proof of Case 3, mutatis mutandis.
	
	Case 4: $x,y$ are both even.
	
	Then $x=2m$ and $y=2n$, where $m,n \in \Z$, and
	\begin{align*}
		xy &= (2m)(2n)\\
		&= 4mn\\
		&= 2(2mn).
	\end{align*}
	This is the definition of an even number.

	Since it is shown that both $x$ and $y$ must both be odd for $xy$ to be odd, we can define $x=2m+1$ and $2n+1$ where $m, n \in \Z$, and

	\begin{align*}
		x^2+y^2 &= (2m+1)^2 + (2n+1)^2\\
		&= (4m^2 + 4m + 1) + (4n^2 + 4n + 1)\\
		&= 4m^2+4n^2+4m+4n+2\\
		&= 2(2m^2+2n^2+2m+2n+1).
	\end{align*}

	This is the definition of an even number.

\end{proof}


\question{3}{Write a proof for the following theorem:}

\begin{theorem}
	There is a student $S$ in MA 240 for whom the following is true: If $S$ eats a chocolate on Valentine's Day, then \textit{everyone} in MA 240 will eat a chocolate on Valentine's Day.
\end{theorem}

\begin{proof}
	

	Consider two cases for the class:

	Case 1: There is at least one person in MA 240 not eating chocolate on Valentine's Day.

	Then this leads to an implication that is vacuously true because from the perspective of the one or more people not eating chocolate, that means the antecedent, "if $S$ eats a chocolate on Valentine's Day," of the implication is false. Since this is false, then the consequent, "\textit{everyone} in MA 240 will eat a chocolate on Valentine's Day," will be true regardless; vacuously true.

	Case 2: Everyone in MA 240 is eating chocolate on Valentine's Day.

	Then this leads to a trivial proof because the consequent, "\textit{everyone} in MA 240 will eat a chocolate on Valentine's Day," is always true, therefore, it does not matter whether the antecedent is true or not making this implication always true.
\end{proof}


	
\end{document}