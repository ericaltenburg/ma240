%=======================02-713 LaTeX template, following the 15-210 template==================
%
% You don't need to use LaTeX or this template, but you must turn your homework in as
% a typeset PDF somehow.
%
% How to use:
%    1. Update your information in section "A" below
%    2. Write your answers in section "B" below. Precede answers for all 
%       parts of a question with the command "\question{n}{desc}" where n is
%       the question number and "desc" is a short, one-line description of 
%       the problem. There is no need to restate the problem.
%    3. If a question has multiple parts, precede the answer to part x with the
%       command "\part{x}".
%    4. If a problem asks you to design an algorithm, use the commands
%       \algorithm, \correctness, \runtime to precede your discussion of the 
%       description of the algorithm, its correctness, and its running time, respectively.
%    5. You can include graphics by using the command \includegraphics{FILENAME}
%
\documentclass[11pt]{article}
\usepackage{amsmath,amssymb,amsthm}
\usepackage{graphicx}
\usepackage[margin=1in]{geometry}
\usepackage{fancyhdr}
\newtheorem{theorem}{Theorem}
\setlength{\parindent}{0pt}
\setlength{\parskip}{5pt plus 1pt}
\setlength{\headheight}{13.6pt}
\newcommand\question[2]{\vspace{.25in}\hrule\textbf{#1}: #2\vspace{.5em}\hrule\vspace{.10in}}
\renewcommand\part[1]{\vspace{.10in}(#1)\par}
\newcommand\algorithm{\vspace{.10in}\textbf{Algorithm: }}
\newcommand\correctness{\vspace{.10in}\textbf{Correctness: }}
\newcommand\runtime{\vspace{.10in}\textbf{Running time: }}
\newcommand{\R}{\mathbb{R}}
\newcommand{\N}{\mathbb{N}}
\newcommand{\Z}{\mathbb{Z}}
\pagestyle{fancyplain}
\lhead{\textbf{\NAME}}
\chead{\textbf{{\COURSE} Homework \HWNUM}}
\rhead{\today}
\begin{document}\raggedright
%Section A==============Change the values below to match your information==================
\newcommand\NAME{Eric Altenburg}  % your name
\newcommand\COURSE{MA-240}
\newcommand\HWNUM{4 Corrections}              % the homework number
%Section B==============Put your answers to the questions below here=======================

% no need to restate the problem --- the graders know which problem is which,
% but replacing "The First Problem" with a short phrase will help you remember
% which problem this is when you read over your homeworks to study.

\textbf{Pledge:} \textit{I pledge my honor that I have abided by the Stevens Honor System.} -Eric Altenburg

\question{1}{Recall that the sequence of Fibonacci numbers is defined as follows: $F_1 = F_2 = 1$, and $F_n = F_{n-1}+F_{n-2}$ when $n > 2$.}

\part{Prove that $F_1^2+F_2^2+ \ldots + F_n^2 = F_nF_{n+1}$ for all $n \in \N$.}
	\begin{proof}
		(Induction)\\
		Base Case: $n = 1$, then
		\begin{align*}
			F_1^2 &= F_1 \cdot F_{2}\\
			1 &= 1 \cdot 1\\
			1 &= 1.
		\end{align*}
		Inductive Hypothesis: Assume $F_1^2+F_2^2+ \ldots + F_n^2 = F_nF_{n+1}$ holds true for some $n$. We wish to prove the identity holds for $n+1$ as well, and we find that
		\begin{align*}
			F_1^2+F_2^2+ \ldots + F_n^2 + F_{n+1}^2 &= F_nF_{n+1} + F_{n+1}^2\\
			&= F_{n+1} \left(F_n + F_{n+1}\right)\\
			&= F_{n+1} F_{n+2}.
		\end{align*}
		This establishes the claim and completes the proof.
	\end{proof}

\part{Let $S_n$, where $n\in \N$, be the set of all $n$-digit binary strings that have no consecutive 1s. For example, $S_3 = \left\{000, 001, 010, 100, 101\right\}.$ Prove that $S_n$ contains exactly $F_{n+2}$ elements.}

	\begin{proof}
		(Induction)\\
		Base Case:\\
		If $n=1$, then $S_1 = \left\{0, 1\right\}$. $F_3 = 2$ and $\left|S_1\right|= 2$.\\
		If $n=2$, then $S_2 = \left\{00, 01, 10\right\}$. $F_4 = 3$ and $\left|S_2\right|= 3$.\\
		Inductive Hypothesis: Assume $S_n$ contains exactly $F_{n+2}$ elements, where $S_n$ is the set of all $n$-digit binary strings that have no consecutive 1s and $n \in \N$. We wish to prove the identity holds for $S_{n+1}$ as well, and we observe that
		\begin{align*}
			\mid S_{n+1} \mid &= \mid S_n \mid + \mid S_{n-1} \mid\\
			&= F_{n+2} + F_{(n-1)+2}\\
			&= F_{n+2} + F_{n+1}\\
			&= F_{n+3}\\
			&= F_{(n+1)+2}.
		\end{align*}
		This establishes the claim and completes the proof.
	\end{proof}

	\newpage

\question{2}{Suppose you draw several straight lines on a piece of paper, thereby dividing the paper into different regions (each line should extend to the edges of the page). Prove that, no matter how your lines are drawn, it is possible to color each region red or blue in such a way that no two adjacent regions have the same color.}

\begin{proof}
	(Induction) Suppose $n$ is the number of straight lines going from edge-to-edge drawn on a piece of paper.\\
	Base Case: $n=1$\\
	There are two regions on the piece of paper, one can be colored red and the other is colored blue. The two regions are adjacent and do not have the same colors.\\
	Inductive Hypothesis: Assume that for a piece of paper with $n$ lines on it, there are no two adjacent regions that have the same color.\\
	We wish to prove that the property holds for $n+1$ lines as well. After drawing the additional line on the paper, suppose one side of the new line containing all that side's regions is called $A$, and the other side of the new line containing all that side's regions is called $B$. We know that all the adjacent regions in side $A$ and all the adjacent regions in side $B$ satisfy the property, however, the adjacent regions where sides $A$ and $B$ meet will have the same color. To make the identity hold true, all the regions' colors on either side $A$ or $B$ can be inverted; for example, all of $A$'s red regions become blue and blue regions become red. Doing this will still ensure that all the old regions will still follow the identity since they did prior to the inversion, because of this, these color changes will not break it. \\


	% and after drawing the additional line on the paper we know that on both sides of the new line, there are no two regions with the same color. However, the property is not currently satisfied as the newly created regions from the new line are the same color and adjacent to each other. 

	
	Having inverted all the regions to one side of the new line, the regions adjacent to each other where sides $A$ and $B$ meet are different colors and the identity holds true still for $n+1$ lines.
\end{proof}











	
\end{document}