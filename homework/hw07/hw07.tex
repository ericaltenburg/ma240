%=======================02-713 LaTeX template, following the 15-210 template==================
%
% You don't need to use LaTeX or this template, but you must turn your homework in as
% a typeset PDF somehow.
%
% How to use:
%    1. Update your information in section "A" below
%    2. Write your answers in section "B" below. Precede answers for all 
%       parts of a question with the command "\question{n}{desc}" where n is
%       the question number and "desc" is a short, one-line description of 
%       the problem. There is no need to restate the problem.
%    3. If a question has multiple parts, precede the answer to part x with the
%       command "\part{x}".
%    4. If a problem asks you to design an algorithm, use the commands
%       \algorithm, \correctness, \runtime to precede your discussion of the 
%       description of the algorithm, its correctness, and its running time, respectively.
%    5. You can include graphics by using the command \includegraphics{FILENAME}
%
\documentclass[11pt]{article}
\usepackage{amsmath,amssymb,amsthm}
\usepackage{graphicx}
\usepackage[margin=1in]{geometry}
\usepackage{fancyhdr}
\newtheorem{theorem}{Theorem}
\setlength{\parindent}{0pt}
\setlength{\parskip}{5pt plus 1pt}
\setlength{\headheight}{13.6pt}
\newcommand\question[2]{\vspace{.25in}\hrule\textbf{#1}: #2\vspace{.5em}\hrule\vspace{.10in}}
\renewcommand\part[1]{\vspace{.10in}(#1)\par}
\newcommand\algorithm{\vspace{.10in}\textbf{Algorithm: }}
\newcommand\correctness{\vspace{.10in}\textbf{Correctness: }}
\newcommand\runtime{\vspace{.10in}\textbf{Running time: }}
\newcommand{\R}{\mathbb{R}}
\newcommand{\N}{\mathbb{N}}
\newcommand{\Z}{\mathbb{Z}}
\pagestyle{fancyplain}
\lhead{\textbf{\NAME}}
\chead{\textbf{{\COURSE} Homework \HWNUM}}
\rhead{\today}
\begin{document}\raggedright
%Section A==============Change the values below to match your information==================
\newcommand\NAME{Eric Altenburg}  % your name
\newcommand\COURSE{MA-240}
\newcommand\HWNUM{7}              % the homework number
%Section B==============Put your answers to the questions below here=======================

% no need to restate the problem --- the graders know which problem is which,
% but replacing "The First Problem" with a short phrase will help you remember
% which problem this is when you read over your homeworks to study.

\textbf{Pledge:} \textit{I pledge my honor that I have abided by the Stevens Honor System.} -Eric Altenburg

\question{1}{Prove that if two sets $A$ and $B$ have the same cardinality, then for any set $C$, the sets $A \times C$ and $B \times C$ have the same cardinality as well.}

\begin{proof}
	Let $n = \mid A \mid$, and since $\mid A \mid = \mid B \mid$, then $n = \mid A \mid = \mid B \mid$. Additionally, let $m = \mid C \mid$. To determine the cardinality of the Cartesian product, you multiply the cardinality of the two sets (i.e. $\mid X \times Y \mid = \mid X \mid \cdot \mid Y \mid$). Therefore, 
	\begin{align*}
		\mid A \times C \mid &= \mid B \times C \mid\\
		\mid A \mid \cdot \mid C \mid &= \mid B \mid \cdot \mid C \mid\\
		n \cdot m &= n \cdot m.
	\end{align*}
	This shows that the cardinality of the two Cartesian products are always equal if $\mid A \mid = \mid B \mid$ which completes the proof.
\end{proof}

\question{2}{Is it true that a set $A$ is countable if and only if there exists a surjection $f : \N \rightarrow A$? (Recall that a surjection is an onto function.) If so, prove it. If not, state and prove a closely related theorem.}

\question{3}{Prove that the infinite strip $S = \{(x,y) \in \R^2 \mid x \in \R \text{ and } 0 \le y \le 1 \}$ and the Cartesian plane $\R^2$ have the same cardinality. \textit{Hint:} You may assume the Schröder-Bernstein theorem.}

\question{4}{Prove the theorem whose proof we didn't complete in class: For any set $A$, we have $\mid A \mid < \mid 2^A \mid$.}



	
\end{document}