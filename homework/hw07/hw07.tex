%=======================02-713 LaTeX template, following the 15-210 template==================
%
% You don't need to use LaTeX or this template, but you must turn your homework in as
% a typeset PDF somehow.
%
% How to use:
%    1. Update your information in section "A" below
%    2. Write your answers in section "B" below. Precede answers for all 
%       parts of a question with the command "\question{n}{desc}" where n is
%       the question number and "desc" is a short, one-line description of 
%       the problem. There is no need to restate the problem.
%    3. If a question has multiple parts, precede the answer to part x with the
%       command "\part{x}".
%    4. If a problem asks you to design an algorithm, use the commands
%       \algorithm, \correctness, \runtime to precede your discussion of the 
%       description of the algorithm, its correctness, and its running time, respectively.
%    5. You can include graphics by using the command \includegraphics{FILENAME}
%
\documentclass[11pt]{article}
\usepackage{amsmath,amssymb,amsthm}
\usepackage{graphicx}
\usepackage[margin=1in]{geometry}
\usepackage{fancyhdr}
\newtheorem{theorem}{Theorem}
\setlength{\parindent}{0pt}
\setlength{\parskip}{5pt plus 1pt}
\setlength{\headheight}{13.6pt}
\newcommand\question[2]{\vspace{.25in}\hrule\textbf{#1}: #2\vspace{.5em}\hrule\vspace{.10in}}
\renewcommand\part[1]{\vspace{.10in}(#1)\par}
\newcommand\algorithm{\vspace{.10in}\textbf{Algorithm: }}
\newcommand\correctness{\vspace{.10in}\textbf{Correctness: }}
\newcommand\runtime{\vspace{.10in}\textbf{Running time: }}
\newcommand{\R}{\mathbb{R}}
\newcommand{\N}{\mathbb{N}}
\newcommand{\Z}{\mathbb{Z}}
\pagestyle{fancyplain}
\lhead{\textbf{\NAME}}
\chead{\textbf{{\COURSE} Homework \HWNUM}}
\rhead{\today}
\begin{document}\raggedright
%Section A==============Change the values below to match your information==================
\newcommand\NAME{Eric Altenburg}  % your name
\newcommand\COURSE{MA-240}
\newcommand\HWNUM{7 Corrections}              % the homework number
%Section B==============Put your answers to the questions below here=======================

% no need to restate the problem --- the graders know which problem is which,
% but replacing "The First Problem" with a short phrase will help you remember
% which problem this is when you read over your homeworks to study.

\textbf{Pledge:} \textit{I pledge my honor that I have abided by the Stevens Honor System.} -Eric Altenburg

\question{1}{Prove that if two sets $A$ and $B$ have the same cardinality, then for any set $C$, the sets $A \times C$ and $B \times C$ have the same cardinality as well.}

\begin{proof}
	% Let $n = |A|$, and since $| A | = | B |$, then $n = | A | = | B |$. Additionally, let $m = | C |$. To determine the cardinality of the Cartesian product, you multiply the cardinality of the two sets (i.e. $| X \times Y | = | X | \cdot | Y |$). Therefore, 
	% \begin{align*}
	% 	| A \times C | &= \,| B \times C |\\
	% 	| A | \cdot | C | &= \,| B | \cdot | C |\\
	% 	n \cdot m &= n \cdot m.
	% \end{align*}
	% This shows that the cardinality of the two Cartesian products are always equal if $| A | = | B |$ which completes the proof.
	Since $|A|=|B|$, there must exist a bijective function $f$ such that $f : A \rightarrow B$, and it is defined as $f(a) \rightarrow b$ where $a \in A$ and $b \in B$. Now to show that $|A \times C| = |B \times C|$, we can create another bijective function $g$ such that $g : A \times C \rightarrow B \times C$, and it is defined as $g(a,c) \rightarrow (f(a), c)$ where $c \in C$. Because this bijective function exists, it is proven that $|A \times C| = |B \times C|$.
\end{proof}


% https://leanprover.github.io/logic_and_proof/the_infinite.html
\question{2}{Is it true that a set $A$ is countable if and only if there exists a surjection $f : \N \rightarrow A$? (Recall that a surjection is an onto function.) If so, prove it. If not, state and prove a closely related theorem.}

The way the theorem is currently written cannot be proven because it is possible for $A$ to be an empty set. Therefore, the theorem should be rewritten to account for this.

\begin{theorem}
	The set $A$ is countable if and only if $A$ is empty or there exists a surjection $f : \N \rightarrow A$.
\end{theorem}

\begin{proof}
	($\Longrightarrow$)\\
	Suppose $A$ is countable. Then this means $A$ is either finite or infinitely countable.\\
	Case 1: $A$ is infinitely countable.\\
	If $A$ is infinitely countable, then there exists a bijective function from $\N$ to $A$. And since it is bijective, then the function from $\N$ to $A$ is also surjective.\\
	Case 2: $A$ is finite and is empty.\\
	The empty set is countable.\\
	Case 3: $A$ is finite and not empty.\\
	Since $A$ is finite and countable, then there is a bijective function for some $n$ where $f:[n] \rightarrow A$ such that $n \ge 1$ and $0 \in [n]$. We can create a surjection between $\N$ and $A$ by defining $g: \N \rightarrow A$ as,
	\begin{equation*}
		g(x) \rightarrow
		\begin{cases}
			f(x) & \text{if $x < n$}\\
			f(0) & \text{else}
		\end{cases}\,.
	\end{equation*}
	The function $g$ simply returns each element in the set $A$, and then it will repeat $f(0)$. This is surjective.\\
	($\Longleftarrow$)\\
	Case 1: $A$ is finite.\\
	If $A$ is finite, then it is countable.\\
	Case 2: $A$ is infinite.\\
	If $A$ is infinite, then it is not empty, and there is a surjection $f: \N \rightarrow A$. To make $f$ bijective, we need to account for the issue of an element in $A$ being enumerated more than once because if it does, then it is not injective. To fix this, we can create another function $g$ that will enumerate $f$'s output and does not include the output that has been repeated. Define $g$ as $g(0) = f(0)$ and for every $i$, $g(i+1) = f(j)$ such that $j$ is the least number in $\N$ and $f(j) \not\in \{g(0), g(1), \ldots, g(j)\}$. The function $g$ is a bijection because for every $i$, $g(i+1) \not\in \{g(0), g(1), \ldots, g(i)\}$ which makes it injective, and since $f$ is surjective, $g$ is as well. This means $A$ countable.  
\end{proof}


\question{3}{Prove that the infinite strip $S = \{(x,y) \in \R^2 \mid x \in \R \text{ and } 0 \le y \le 1 \}$ and the Cartesian plane $\R^2$ have the same cardinality. \textit{Hint:} You may assume the Schröder-Bernstein theorem.}

\begin{proof}
	Assuming the Schröder-Bernstein theorem, we can create an injective function $f:S \rightarrow \R^2$ by defining $f$ as $f(x,y) \rightarrow (x,y)$ where any pair $(x,y) \in S$ will give itself in $\R^2$. This shows that $| S | \le | \R^2 |$. To create an injective function $g : \R^2 \rightarrow S$, we can define $g$ as,
	\begin{equation*}
		g(x,y) \rightarrow
		\begin{cases}
			\left(x,\left(\frac{1}{2} + 2^{-y-1}\right)\right) , & y > 0\\
			\left(x,\frac{1}{2}\right), & y = 0\\
			\left(x, 2^{y-1}\right), & y < 0
		\end{cases}\,.
	\end{equation*}
	This shows that $| \R^2 | \le | S |$, and because of the Schröder-Bernstein theorem, we must have that $| S | = | \R^2 |$.
\end{proof}

% cantor's theorem
\question{4}{Prove the theorem whose proof we didn't complete in class: For any set $A$, we have $| A | < | 2^A |$.}

\begin{proof}
	(Contradiction) First we can construct a surjective function $f: A \rightarrow 2^A$ such that $f(a) \rightarrow a$ where $a \in A$. This implies that $|A| \le |2^A|$, however, to prove that the cardinality of $A$ is strictly less than the cardinality of $2^A$ we must show that a bijection does not exist from $A$ to $2^A$. Assuming the opposite and that such a function exists, let the function be $g : A \rightarrow 2^A$. Using this, we can construct the set $S=\{x \in A \mid x \not\in g(x)\}$, note that $S \subseteq A$. Because $S \subseteq 2^A$ and our function $g$ is also surjective, this must mean that $g(y) \in S$ for some $y \in A$. This is a contradiction because if we have some $y \in S$, then that means $y \not \in g(y)$ but like we stated earlier, $g(y) \in S$. Additionally, if $y \not \in S$, then this means $y \in g(y)$ but $g(y) \in S$. So regardless of $y$'s membership with $S$ there is a contradiction which must mean that no bijective function $g$ exists. Therefore, $|A| < |2^A|$.
\end{proof}

	
\end{document}