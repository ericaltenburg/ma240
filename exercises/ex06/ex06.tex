%=======================02-713 LaTeX template, following the 15-210 template==================
%
% You don't need to use LaTeX or this template, but you must turn your homework in as
% a typeset PDF somehow.
%
% How to use:
%    1. Update your information in section "A" below
%    2. Write your answers in section "B" below. Precede answers for all 
%       parts of a question with the command "\question{n}{desc}" where n is
%       the question number and "desc" is a short, one-line description of 
%       the problem. There is no need to restate the problem.
%    3. If a question has multiple parts, precede the answer to part x with the
%       command "\part{x}".
%    4. If a problem asks you to design an algorithm, use the commands
%       \algorithm, \correctness, \runtime to precede your discussion of the 
%       description of the algorithm, its correctness, and its running time, respectively.
%    5. You can include graphics by using the command \includegraphics{FILENAME}
%
\documentclass[11pt]{article}
\usepackage{amsmath,amssymb,amsthm}
\usepackage{graphicx}
\usepackage[margin=1in]{geometry}
\usepackage{fancyhdr}
\newtheorem{theorem}{Theorem}
\setlength{\parindent}{0pt}
\setlength{\parskip}{5pt plus 1pt}
\setlength{\headheight}{13.6pt}
\newcommand\question[2]{\vspace{.25in}\hrule\textbf{#1}: #2\vspace{.5em}\hrule\vspace{.10in}}
\renewcommand\part[1]{\vspace{.10in}(#1)\par}
\newcommand\algorithm{\vspace{.10in}\textbf{Algorithm: }}
\newcommand\correctness{\vspace{.10in}\textbf{Correctness: }}
\newcommand\runtime{\vspace{.10in}\textbf{Running time: }}
\newcommand{\R}{\mathbb{R}}
\newcommand{\N}{\mathbb{N}}
\newcommand{\Z}{\mathbb{Z}}
\pagestyle{fancyplain}
\lhead{\textbf{\NAME}}
\chead{\textbf{{\COURSE} Lesson \HWNUM \text{ }Exercises}}
\rhead{\today}
\begin{document}\raggedright
%Section A==============Change the values below to match your information==================
\newcommand\NAME{Eric Altenburg}  % your name
\newcommand\COURSE{MA-240}
\newcommand\HWNUM{6}              % the homework number
%Section B==============Put your answers to the questions below here=======================

% no need to restate the problem --- the graders know which problem is which,
% but replacing "The First Problem" with a short phrase will help you remember
% which problem this is when you read over your homeworks to study.

\textbf{Pledge:} \textit{I pledge my honor that I have abided by the Stevens Honor System.} -Eric Altenburg

\question{1}{In 1202, the Italian mathematician Leonardo Pisano published the seminal work Liber Abaci, which among other things introduced the following sequence of numbers: $1, 1, 2, 3, 5, 8, 13, 21, 34, 55, 89, 144, \cdots$. This sequence admits a \textit{recursive definition}, namely $F_1 = F_2 = 1$, and for all $n > 2$ we put $F_n = F_{n-1} + F_{n-2}$. For this reason, it is possible to prove many things about the sequence ${F_n}_{n \in \N}$ using induction.}

\part{Leonardo Pisano is better known by another name. What is it?}
	Fibonacci

\part{Use induction to prove that the numbers ${F_n}_{n \in \N}$ satisfy the condition $F^2_{n+1}-F_{n+1}F_n-F^3_n = (-1)^n$ for all $n \in \N$}
	\begin{proof}
		(Induction)\\
		Base Case: $n=1$\\	
		$1^2 - 1\cdot1-1^2 = 1-1-1 = -1 = -1$\\
		Inductive Hypothesis: Assume $F^2_{n+1}-F_{n+1}F_n-F^3_n = (-1)^n$ for some $n \in \N$.\\
		Inductive Step: 
		\begin{align*}
			F^2_{n+2}-F_{n+2}F_{n+1}-F^3_{n+1} &= (-1)^{n+1}\\
			F_{n+1} \cdot F_{n+1} \cdot F_n\cdot F_n + F_{n+1} \cdot F_{n+1} \cdot F_n - F_{n+1} \cdot F_{n+1} \cdot F_{n+1} &= (-1)^{n+1}
		\end{align*}
	\end{proof}

	I'm stuck here in the proof. I think I understand the basic idea where you try to sub in certain terms for it's definition the Fibonacci sequence of $F_n = F_{n-2} + F_{n-1}$ however I can't seem to find the correct sequence of replacements to get the LHS and RHS in such a way that multiplying both sides by $-1$ will turn the RHS and LHS back into the inductive hypothesis.

\part{Although the above condition may look idiosyncratic, it has a beautiful consequence, which we will discuss in class. Any idea what it is?}
	Based on where I wanted to take my proof, that through the the use of recursion it is not necessary to prove explicitly that $n+1$ is true but instead through substitutions you can instead prove that the inductive hypothesis is true.


\question{2}{We all know that $(1+2+ \cdots + n)^2 \ne 1^2 + 2^2 + \cdots + n^2$. To believe otherwise would be to fall for one of the most basic algebra mistakes in the book! But did you know that $(1+2+\cdots+n)^2 = 1^3 + 2^3 + \cdots + n^3$ for all $n \in \N$? Prove it using induction!}

\begin{proof}
	(Induction)\\
	Base Case: $n=1$\\
	$1^2 = 1^3$\\
	Inductive Hypothesis: Assume that $(1+2+\cdots+n)^2 = 1^3 + 2^3 + \cdots + n^3$ is true for some $n \in \N$.\\
	Inductive Step: Rewrite the LHS: $\left(\frac{n(n+1)}{2} \right)^2 = 1^3 + 2^3 + \cdots + n^3$. Now consider: 

	\begin{align*}
		\left(\frac{n(n+1)}{2} \right)^2 + (n+1)^3 &= 1^3 + 2^3 + \cdots + n^3 + (n+1)^3\\
		\frac{n^2(n+1)^2 + 4(n+1)^3}{4} &= 1^3 + 2^3 + \cdots + n^3 + (n+1)^3\\
		\frac{n^4 + 6n^3 + 13n^2 + 12n + 4}{4} &= 1^3 + 2^3 + \cdots + n^3 + (n+1)^3\\
		\frac{(n+1)(n+1)(n+2)(n+2)}{4} &= 1^3 + 2^3 + \cdots + n^3 + (n+1)^3\\
		\left(\frac{(n+1)(n+2)}{2}\right)^2 &= 1^3 + 2^3 + \cdots + n^3 + (n+1)^3
	\end{align*}

	The holds for $n+1$, therefore by induction it holds for each $n \in \N$.
\end{proof}

\question{3}{In the mysterious country of Summandia, there are only three and five dollar bills. This doesn't bother the natives much, though, because they can use these bills to make exact change for any while dollar amount of \$8 or more. Use strong induction to prove that this is indeed the case.}

\begin{proof}
	(Strong Induction) Suppose $P(n)$ is the statement that an amount of \$$n$ can be made with 3 and 5 dollar bills.

	Base Case: $P(8) = 3 \cdot 1 + 5 \cdot 1$\\
	$P(9) = 3 \cdot 3 + 5 \cdot 0$\\
	$P(10) = 3 \cdot 0 + 5 \cdot 2$\\
	Inductive Hypothesis: Assume $P(n)$ is true for $8 \le n \le k$ where $k \ge 10$ because this is the minimum increment from the base amount of \$8 that can be made with \$3.\\
	Inductive Step: $k+1 = (k-2)+3$ and we know that since $k \ge 10$, $k \ge 8$ which means it can be made from 3 and 5 dollar bills as assumed from the inductive hypothesis. So since $P(k-2)$ is true, we can add a 3 dollar bill to the $P(k-2)$ amount which would be $P(k+1)$ which is true.
\end{proof}



	
\end{document}