%=======================02-713 LaTeX template, following the 15-210 template==================
%
% You don't need to use LaTeX or this template, but you must turn your homework in as
% a typeset PDF somehow.
%
% How to use:
%    1. Update your information in section "A" below
%    2. Write your answers in section "B" below. Precede answers for all 
%       parts of a question with the command "\question{n}{desc}" where n is
%       the question number and "desc" is a short, one-line description of 
%       the problem. There is no need to restate the problem.
%    3. If a question has multiple parts, precede the answer to part x with the
%       command "\part{x}".
%    4. If a problem asks you to design an algorithm, use the commands
%       \algorithm, \correctness, \runtime to precede your discussion of the 
%       description of the algorithm, its correctness, and its running time, respectively.
%    5. You can include graphics by using the command \includegraphics{FILENAME}
%
\documentclass[11pt]{article}
\usepackage{amsmath,amssymb,amsthm}
\usepackage{graphicx}
\usepackage[margin=1in]{geometry}
\usepackage{fancyhdr}
\newtheorem{theorem}{Theorem}
\setlength{\parindent}{0pt}
\setlength{\parskip}{5pt plus 1pt}
\setlength{\headheight}{13.6pt}
\newcommand\question[2]{\vspace{.25in}\hrule\textbf{#1}: #2\vspace{.5em}\hrule\vspace{.10in}}
\renewcommand\part[1]{\vspace{.10in}(#1)\par}
\newcommand\algorithm{\vspace{.10in}\textbf{Algorithm: }}
\newcommand\correctness{\vspace{.10in}\textbf{Correctness: }}
\newcommand\runtime{\vspace{.10in}\textbf{Running time: }}
\newcommand{\R}{\mathbb{R}}
\newcommand{\N}{\mathbb{N}}
\newcommand{\Z}{\mathbb{Z}}
\newtheorem{lemma}[theorem]{Lemma}
\pagestyle{fancyplain}
\lhead{\textbf{\NAME}}
\chead{\textbf{{\COURSE} Lesson \HWNUM \text{ }Exercises}}
\rhead{\today}
\begin{document}\raggedright
%Section A==============Change the values below to match your information==================
\newcommand\NAME{Eric Altenburg}  % your name
\newcommand\COURSE{MA-240}
\newcommand\HWNUM{8/9}              % the homework number
%Section B==============Put your answers to the questions below here=======================

% no need to restate the problem --- the graders know which problem is which,
% but replacing "The First Problem" with a short phrase will help you remember
% which problem this is when you read over your homeworks to study.

\textbf{Pledge:} \textit{I pledge my honor that I have abided by the Stevens Honor System.} -Eric Altenburg

\question{1}{Try your hand at the following lemma.}

\begin{lemma}
	Let $R$ be an equivalence relation on a nonempty set $A$, and let $a,b \in A$. Then $\left[a\right]=\left[b\right]$ if and only if $(a,b)\in R$.
\end{lemma}

\begin{proof}
	($\Rightarrow$) Assume $[a]=[b]$. Since we know that $R$ is an equivalence relation, it is reflexive. Therefore, $[a] = [a]$ and from our original assumption we say that $[a]=[b]$, so it shows that $a,b \in R$.\\
	{$\Leftarrow$} Assume that $(a,b) \in R$. Let $x \in [a]$, and since $(a,b) \in R$, due to transitivity $(x,b) \in R$ so $x \in [b]$. Let $y \in [b]$, and since $(a,b) \in R$, due to symmetry we have $(y,a) \in R$ so $y \in [a]$. Therefore, it is proven.
\end{proof}



\question{2}{Recall that a partition of a set $A$ is a collection of nonempty pairwise-disjoint subsets of $A$ whose union is all of $A$. Now try your hand at proving the following theorem.}

\begin{theorem}
	If $R$ is an equivalence relation on a nonempty set $A$, then the set $\Pi=\{[a]\}_{a\in A}$ of equivalence classes of elements of $A$ is partition of $A$.
\end{theorem}	

I'm not sure where to really begin when trying to prove this and cannot form it into a formal proof, but loosely the logic I would try to follow would be to show that because $\Pi$ is set of equivalence classes, we can use it to show that they form sets whose union will form $A$.


\question{3}{Given integers $a,b \in \Z$ and an integer $n > 1$, we say that $A$ and $B$ are congruent modulo $n$ written $a\equiv b\,\,(\text{mod}\,n)$ if $n$ divides $a-b$.}

\part{Prove that for any $n>1$, being congruent modulo $n$ is an equivalence relation.}

\begin{proof}
	To show that for any $n>1$, being congruent modulo $n$ is an equivalence relation we need to show that congruency is both reflexive, symmetric, and transitive.\\
	(Reflexive) We can say $a-a=0x$ where $x$ is some integer. Then it is reasonable to say that $a \equiv a(\text{mod }n)$.\\
	(Symmetric) If we assume that $a\equiv b\,\,(\text{mod}\,n)$, then it stands that $a-b=nx$ where $x$ is some integer. Then we can multiply both sides by -1 and obtain $b-a=n(-x)$ which is the same as saying $b\equiv a\,\,(\text{mod}\,n)$,\\
	(Transitive) Assume we have $a\equiv b\,\,(\text{mod}\,n)$ and $b\equiv c\,\,(\text{mod}\,n)$. Then we have the two equations $a-b=nx$ for some integer $x$ and $b-c=ny$ for some integer $y$. When combining the equations together we get $(a-b)+(b-c) = nx + ny \Rightarrow a-c=n(x+y)$ and since $x,y \in \Z$ we can say that $a\equiv c\,\,(\text{mod}\,n)$.
\end{proof}

\part{Describe the equivalence classes of congruence modulo 3.}
 	Since we are going up to 3 for this modulo there are only three sets of integers that satisfy this congruence. The first is the set of all integers which are congruent to 1 mod 3. The second is the set of all integers which are congruent to 2 mod 3. Finally, the third is the set of all integers which are congruent to 3 mod 3.




	
\end{document}