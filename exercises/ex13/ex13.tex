%=======================02-713 LaTeX template, following the 15-210 template==================
%
% You don't need to use LaTeX or this template, but you must turn your homework in as
% a typeset PDF somehow.
%
% How to use:
%    1. Update your information in section "A" below
%    2. Write your answers in section "B" below. Precede answers for all 
%       parts of a question with the command "\question{n}{desc}" where n is
%       the question number and "desc" is a short, one-line description of 
%       the problem. There is no need to restate the problem.
%    3. If a question has multiple parts, precede the answer to part x with the
%       command "\part{x}".
%    4. If a problem asks you to design an algorithm, use the commands
%       \algorithm, \correctness, \runtime to precede your discussion of the 
%       description of the algorithm, its correctness, and its running time, respectively.
%    5. You can include graphics by using the command \includegraphics{FILENAME}
%
\documentclass[11pt]{article}
\usepackage{amsmath,amssymb,amsthm}
\usepackage{graphicx}
\usepackage[margin=1in]{geometry}
\usepackage{fancyhdr}
\newtheorem{theorem}{Theorem}
\setlength{\parindent}{0pt}
\setlength{\parskip}{5pt plus 1pt}
\setlength{\headheight}{13.6pt}
\newcommand\question[2]{\vspace{.25in}\hrule\textbf{#1}: #2\vspace{.5em}\hrule\vspace{.10in}}
\renewcommand\part[1]{\vspace{.10in}(#1)\par}
\newcommand\algorithm{\vspace{.10in}\textbf{Algorithm: }}
\newcommand\correctness{\vspace{.10in}\textbf{Correctness: }}
\newcommand\runtime{\vspace{.10in}\textbf{Running time: }}
\newcommand{\R}{\mathbb{R}}
\newcommand{\N}{\mathbb{N}}
\newcommand{\Z}{\mathbb{Z}}
\pagestyle{fancyplain}
\lhead{\textbf{\NAME}}
\chead{\textbf{{\COURSE} Lesson \HWNUM \text{ }Exercises}}
\rhead{\today}
\begin{document}\raggedright
%Section A==============Change the values below to match your information==================
\newcommand\NAME{Eric Altenburg}  % your name
\newcommand\COURSE{MA-240}
\newcommand\HWNUM{13}              % the homework number
%Section B==============Put your answers to the questions below here=======================

% no need to restate the problem --- the graders know which problem is which,
% but replacing "The First Problem" with a short phrase will help you remember
% which problem this is when you read over your homeworks to study.

\textbf{Pledge:} \textit{I pledge my honor that I have abided by the Stevens Honor System.} -Eric Altenburg

\question{1}{The statement of the fundamental theorem of arithmetic given above is somewhat vague. What is a prime number? What is a product of primes? And what does it mean for such a product to be unique? Explain.}

A natural number $p$ is prime if its only divisors are 1 and $p$, and $p \ne 1$.\\
A product of primes can be written as $n = p_1 \cdot p_2 \cdot p_3 \cdot \ldots \cdot p_k$ where $n,k \in \N$\\
For a product to be unique, it means that when writing the product of primes as $n = p_1 \cdot p_2 \cdot p_3 \cdot \ldots \cdot p_k$ where $n,k \in \N$, no two or more $p_k$'s can be the same number. For example, $20 = 2 \cdot 2 \cdot 5 = 2^2 \cdot 5$ would not be unique as there are two 2's.


\question{2}{Our first goal will be to prove the existence of a factorization into primes. Write down a strategy for doing so, and be prepared to explain it in class.}

We can assume that there exists a number that does not have a factorization into primes. Let $n$ be the smallest such number that cannot be factored into primes. Since $n$ is not prime, then it is composite and can be factored as such $n = a \cdot b$ such that $1 < a$ and $b < n$. However, since we assumed that $n$ was the smallest such number that was unable to be factored into primes, then $a$ and $b$ must be products of primes. With this, $a \cdot b$ is a product of primes and so $n$ is a product of primes as well. This contradicts the initial assumption.

% The idea behind finding the factorization of primes can be summed up by the following steps on a number $n \in \N$:\\
% If $n$ is prime, then the factorization exists.\\
% If $n$ is not prime, then $n=a \cdot b$ where $1 < a$ and $b < n$. If $a$ and $b$ are both prime, then we have the factorization.\\
% If either $a$ or $b$ is not a prime number, then they should be broken up in the same way that $n$ was. For example, assume $a$ and $b$ are not prime, then $n = a\cdot b = (a' \cdot b') \cdot (a'' \cdot b'')$. This pattern of continually splitting up non-prime factors continues until both $a$ and $b$ are prime. Once this is achieved, the factorization of primes is achieved.


\question{3}{Our next goal will be to prove the uniqueness of a factorization into primes, which will entail several steps. For each of these steps, write down another proof strategy, and be prepared to explain it in class.}

\part{Step 1: Prove Bézout's theorem, namely that if two integers $a$ and $b$ are coprime, then there exist integers $x$ and $y$ such that $ax+by=1$.}

If $a$ and $b$ are coprime, then we can use the idea that $\gcd(a,b) = 1$ to show that $\gcd(a,b) = ax+by$ which would show that $x$ and $y$ exist. Assume there is a common divisor $c$ for $a$ and $b$. $c$ would then divide $ax$ and $by$ or $ax + by$ which we know to be 1. That means $c$ divides 1 so $c=1$. Since $c=1$, this shows that $x$ and $y$ exist.

\part{Step 2: Prove Euclid's lemma, namely that if $P$ is prime and $p \mid ab$, where $a,b \in \Z$, then $p \mid a$ or $p \mid b$}

We can use Bézout's theorem for this where $ax + by = 1$ and $a,b$ are relatively prime. In the context of this problem, let's assume $n$ and $a$ are relatively prime numbers and $n \mid ab$. Then with Bézout's theorem, we can say $nx + ay = 1$. Multiplying both sides by $b$ would give $nxb + ayb = b$. This means that $n$ divides $(nxb)$ and it divides $(ayb)$ because of the assumption, and because of this, that means $b$ is also divisible by $n$. This proves that $n \mid b$ which is the lemma.

\part{Step 3: Prove uniqueness with the help of the previous two results.}

Assume we have the smallest number that can be written as two distinct prime factorizations, call it $s$. Then $s = p_1 \cdot p_2 \cdot \ldots \cdot p_m = q_1 \cdot q_2 \cdot \ldots \cdot q_n$. From Euclid's lemma, we know that the prime number $p_1 \mid q_1$ or $p_1 \mid q_2 \cdot \ldots \cdot q_n$. So we can then say that $p_1 = q_k$ for some $k$. So by removing these items from the equation we now have $s' = p_2 \cdot \ldots \cdot p_m = q_1 \cdot q_2 \cdot \ldots \cdot q_n$ but with some $q_k$ missing in the second factorization as well. Because these were removed, we are left with another number $s'$ where $s' < s$ which contradicts the initial assumption, so no such smallest number can exist proving uniqueness.

	
\end{document}