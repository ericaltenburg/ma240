%=======================02-713 LaTeX template, following the 15-210 template==================
%
% You don't need to use LaTeX or this template, but you must turn your homework in as
% a typeset PDF somehow.
%
% How to use:
%    1. Update your information in section "A" below
%    2. Write your answers in section "B" below. Precede answers for all 
%       parts of a question with the command "\question{n}{desc}" where n is
%       the question number and "desc" is a short, one-line description of 
%       the problem. There is no need to restate the problem.
%    3. If a question has multiple parts, precede the answer to part x with the
%       command "\part{x}".
%    4. If a problem asks you to design an algorithm, use the commands
%       \algorithm, \correctness, \runtime to precede your discussion of the 
%       description of the algorithm, its correctness, and its running time, respectively.
%    5. You can include graphics by using the command \includegraphics{FILENAME}
%
\documentclass[11pt]{article}
\usepackage{amsmath,amssymb,amsthm}
\usepackage{graphicx}
\usepackage[margin=1in]{geometry}
\usepackage{fancyhdr}
\newtheorem{theorem}{Theorem}
\setlength{\parindent}{0pt}
\setlength{\parskip}{5pt plus 1pt}
\setlength{\headheight}{13.6pt}
\newcommand\question[2]{\vspace{.25in}\hrule\textbf{#1}: #2\vspace{.5em}\hrule\vspace{.10in}}
\renewcommand\part[1]{\vspace{.10in}\textbf{(#1)}\par}
\newcommand\algorithm{\vspace{.10in}\textbf{Algorithm: }}
\newcommand\correctness{\vspace{.10in}\textbf{Correctness: }}
\newcommand\runtime{\vspace{.10in}\textbf{Running time: }}
\newcommand{\R}{\mathbb{R}}
\newcommand{\N}{\mathbb{N}}
\newcommand{\Z}{\mathbb{Z}}
\pagestyle{fancyplain}
\lhead{\textbf{\NAME}}
\chead{\textbf{{\COURSE} Lesson \HWNUM \text{ }Exercises}}
\rhead{\today}
\begin{document}\raggedright
%Section A==============Change the values below to match your information==================
\newcommand\NAME{Eric Altenburg}  % your name
\newcommand\COURSE{MA-240}
\newcommand\HWNUM{2}              % the homework number
%Section B==============Put your answers to the questions below here=======================

% no need to restate the problem --- the graders know which problem is which,
% but replacing "The First Problem" with a short phrase will help you remember
% which problem this is when you read over your homeworks to study.

\textbf{Pledge:} \textit{I pledge my honor that I have abided by the Stevens Honor System.} -Eric Altenburg

\question{1}{Show that the contrapositive of an implication is logically equivalent to the implication itself.}

The implication $P \Rightarrow Q$ is logically equivalent to $\lnot P \lor Q$. By using this, we can take the contrapositive implication and apply the same steps:

$\lnot Q \Rightarrow \lnot P \equiv Q \lor \lnot P$ 

Which is essentially the same as $\lnot P \lor Q$ with the order simply switched. 

\question{2}{Prove that for an integer $n$, if $n^2$ is even, then $n$ is even.}

\begin{proof}
	The way the current theorem is formatted cannot be proven with a direct proof, instead, it can be proven with a contrapositive. It can then be rewritten as:

	If $n$ is not even, then $n^2$ is not even. 

	We can write an odd number as $n=2k+1$ where k is some integer and substitute that $n$ into $n^2$ to get:

	\begin{align*}
		n^2 &= (2k+1)^2\\
		&= 4k^2 + 4k + 1\\
		&= 2(2k^2 + 2k) + 1
	\end{align*}

	Since we know that the contents of the parenthesis $(2k^2 + 2k)$ will be another integer, then we can let $x = 2k^2 + 2k$. Now the final line of the equation can be rewritten as $2x + 1$ which is the definition of an odd number.
\end{proof}

\question{3}{Let $x \in \R$. Prove that if $x^5-x^4+7x^3-x^2+5x-8 \ge 0$, then $x \ge 0$.}

\begin{proof}
	Once again, proving this theorem as it stands is difficult, so we can rewrite it as a contrapositive:

	Let $x \in \R$. If $x < 0$, then $x^5-x^4+7x^3-x^2+5x-8 < 0$.

	We can rearrange the even and odd powers of the second proposition so it reads $x^5 +7x^3+5x < x^4 +x^2+8$. Because we know that $x < 0$, this inequality will always hold, therefore the original proposition ($x^5-x^4+7x^3-x^2+5x-8 \ge 0$) will not hold if $x \ge 0$.
\end{proof}

\newpage

\question{4}{Prove the following statements. Decide whether a direct proof or providing the contrapositive is more appropriate.}

\begin{theorem}
	If $n$ is odd, then $8$ divides $n^2-1$.
\end{theorem}

\begin{proof}
	(Direct because it is easier)

	Given that $n$ is odd, we can let $n=2k+1$ where $k$ is some integer. Substituting this into $n^2-1$ gives:

	\begin{align*}
		n^2-1 &= (2k+1)^2 - 1\\
		&= 4k^2 + 4k +1 - 1\\
		&= 4k^2 + 4k\\
		&= 4k(k+1)
	\end{align*}

	We have $4 \cdot k \cdot (k+1)$, we know $k \cdot (k+1)$ is an even number because one of the numbers is even. Because of this we can rewrite $k \cdot (k+1) = 2x$ where x is some integer. This gives the final equation $4k(k+1) = 4 \cdot 2x = 8x$ which is divisible by 8.
\end{proof}

\begin{theorem}
	If $n \in \Z$, then $4$ does not divide $n^2-3$.
\end{theorem}

\begin{proof}
	(Contrapositive) The implication is rewritten as:

	If 4 divides $n^2-3$, then $n \in \Z$.

	For $n^2-3$ to be divisible there needs to be a factor of 4, so it can be rewritten as $n^2-3 = 4(\frac{n^2-3}{4})$. In this form there needs to be some sort of decimal to have this number be divisible by 4 and because of this requirement, $n$ cannot be an integer.
\end{proof}






	
\end{document}